\section[Installation]{Installation}
\label{sec:installation}
\addcontentsline{toc}{section}{\thesection. Installation}


\subsection{Installing}

The \pkg{pbdZMQ} package requires an installation of the ZeroMQ library.  So 
before we may discuss particulars of installing the \proglang{R} package, we 
take a moment here to describe the various ways in which you may install ZeroMQ 
itself.  For convenience, we distribute with the package a distribution of 
ZeroMQ, although if you have access to a system installation, that may be 
preferable.  We separate installation of ZeroMQ into 3 cases:
\begin{enumerate}
  \item system package manager, such as the \code{libzmq} and 
  \code{libzmq-dev} packages in Debian-derived systems,
  \item \pkg{pbdZMQ}'s internal ZeroMQ library (4.1.0 rc1), or
  \item an external ZeroMQ library (4.1.0 rc1 or later).
\end{enumerate}


\paragraph{With System Package Manager}

This method is perhaps the easiest when a package managing system is available, 
such as on Linux, and returns the locations of the \code{libzmq} include and 
library paths via:
\begin{itemize}
\item \code{pkg-config --variable=includedir libzmq} and
\item \code{pkg-config --variable=libsdir libzmq}.
\end{itemize}
In this setup, installation is very straightforward.  From a shell, you can 
execute:
\begin{Command}
R CMD INSTALL pbdZMQ_0.1-0.tar.gz
\end{Command}



\paragraph{Using Internal ZeroMQ}

This method uses the ZeroMQ library bundled with \pkg{pbdZMQ}, and should be 
fairly simple.  This method has been successfully tested under Linux, Mac OSX, 
and Windows system. Solaris has been tested with no success. FreeBSD is 
not tested yet.

Installation in this way can be simply done by adding the configure argument 
\code{--enable-internal-zmq}.  In practice, this might look something like:
\begin{Command}
R CMD INSTALL pbdZMQ_0.1-0.tar.gz \
  --configure-args="--enable-internal-zmq"
\end{Command}


\paragraph{Using External ZeroMQ}

This method assumes you have built your own ZeroMQ library somewhere, or 
perhaps one is offered to you by your system administrator.  In any event, 
this method is only tested under Linux systems.  As with the previous method, 
we were unsuccessful in our attempts to build on Solaris.

To build ZeroMQ yourself, you might do something like the following:
\begin{Command}
./configure \
   --prefix=/usr/local/zmq \
   --enable-shared=yes \
   --with-poller=select \
   --without-documentation \
   --without-libsodium
make -j 4
make install
\end{Command}
which will install the library to \code{/usr/local/zmq/} where
\code{/usr/local/zmq/include/} will have the header file \code{zmq.h} and
\code{/usr/local/zmq/lib/} will have the shared library file \code{libzmq.so}.

With an external ZeroMQ available, we can install \pkg{pbdZMQ} via:
\begin{Command}
R CMD INSTALL pbdZMQ_0.1-0.tar.gz \
  --configure-vars="ZMQ_INCLUDE='-I/usr/local/zmq/include' \
                    ZMQ_LDFLAGS='-L/usr/local/zmq/lib -lzmq'"
\end{Command}



\subsection{Testing the Installation}

To make sure that \pkg{pbdZMQ} is installed correctly, one may run a simple 
``hello world'' test from {\em one} terminal to test the library as follows:
\begin{Command}
Rscript -e "demo(hwserver,'pbdZMQ',ask=F,echo=F)" &
Rscript -e "demo(hwclient,'pbdZMQ',ask=F,echo=F)"
\end{Command}
This will run 5 iterations of sending and receiving 'Hello World' messages 
between two instances (simple server and client).



\subsection{Polling System}

Note that one may want to use different polling system provided by 
the ZeroMQ library.  By default, the \code{select} method is used in 
\pkg{pbdZMQ} for Linux, Windows, and Mac OSX. However, users may want to use 
\code{autodetect} or try others for better polling. Currently, the options as 
given by ZeroMQ may be \code{kqueue}, \code{epoll}, \code{devpoll}, 
\code{poll}, or \code{select} depending on libraries and system. You may set 
the polling method at compile time via:
\begin{Command}
R CMD INSTALL pbdZMQ_0.1-0.tar.gz \
  --configure-vars="ZMQ_POLLER='autodetect'"
\end{Command}
See the ZeroMQ manual for more details.