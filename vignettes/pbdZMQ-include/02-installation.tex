\section[Installation]{Installation}
\label{sec:installation}
\addcontentsline{toc}{section}{\thesection. Installation}

There are three ways to install \pkg{pbdZMQ}
\begin{enumerate}
\item with system \code{libzmq} and \code{libzmq-dev} packages
      (as \code{pkg-config} is available),
\item with internal ZeroMQ library (4.1.0 rc1), or
\item with external ZeroMQ library (newer version).
\end{enumerate}
The first way is easiest when a package managing system is available and
produces both
\begin{itemize}
\item \code{pkg-config --variable=includedir libzmq} and
\item \code{pkg-config --variable=libsdir libzmq}.
\end{itemize}
The second way is tested under Linux, Mac OSX, and Windows systems, while
the third way is only tested under Linux system. Solaris has been tested
without any success in neither ways. FreeBSD is not tested yet.

Installation with system \code{libzmq} and \code{libzmq-dev} packages:
it can be simply done with
\begin{Command}
R CMD INSTALL pbdZMQ_0.1-0.tar.gz
\end{Command}

Installation with internal ZeroMQ library:
it can be simply done with
\begin{Command}
R CMD INSTALL pbdZMQ_0.1-0.tar.gz \
  --configure-args="--enable-internal-zmq"
\end{Command}

Example steps for installing with external ZeroMQ library:
\begin{itemize}
\item The minimum steps for installing \pkg{zeromq-4.1.2.tar.gz} is
\begin{Command}
./configure \
   --prefix=/usr/local/zmq \
   --enable-shared=yes \
   --with-poller=select \
   --without-documentation \
   --without-libsodium
make -j 4
make install
\end{Command}
which will install the library to \code{/usr/local/zmq/} where
\code{/usr/local/zmq/include/} will have the header file \code{zmq.h} and
\code{/usr/local/zmq/lib/} will have the shared library file \code{libzmq.so}.

\item Therefore, we can install \pkg{pbdZMQ} as next:
\begin{Command}
R CMD INSTALL pbdZMQ_0.1-0.tar.gz \
  --configure-vars="ZMQ_INCLUDE='-I/usr/local/zmq/include' \
                    ZMQ_LDFLAGS='-L/usr/local/zmq/lib -lzmq'"
\end{Command}

\end{itemize}

Suppose \pkg{pbdZMQ} is installed correctly. One may run the next from
{\em one} terminal to test the library.
\begin{Command}
Rscript -e "demo(hwserver,'pbdZMQ',ask=F,echo=F)" &
Rscript -e "demo(hwclient,'pbdZMQ',ask=F,echo=F)"
\end{Command}
This will run 5 iterations to send and receive 5 times of
'Hello World' messages between two instances (simple server and client).

Note that one may want to use different polling system provided by 
ZeroMQ library.
By default \code{select} method is used in \pkg{pbdZMQ} for Linux, Windows, and
Mac OSX. However, users may want to use \code{autodetect} or try others for
better polling. Currently, the options as given by ZeroMQ may be \code{kqueue},
\code{epoll}, \code{devpoll}, \code{poll}, or \code{select} depending on
libraries and system. The next may be used to change the polling method:
\begin{Command}
R CMD INSTALL pbdZMQ_0.1-0.tar.gz \
  --configure-vars="ZMQ_POLLER='autodetect'"
\end{Command}

